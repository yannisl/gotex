\chapter{Json}


\subsection{Selectors for Json}

If you are familiar with CSS, then you know what selectors are:

\begin{verbatim}
{
  "title": "Java 4-ever",
  "url": "http://www.youtube.com/watch?v=H7QVITAWdBQ",
  "actors": [
    {
      "name": "Scala Johansson",
      "character": "A"
    },
    {
      "name": "William Windows",
      "character": "B"
    },
    {
      "name": "Eddie Larrison",
      "character": "C"
    },
    {
      "name": "Mona Lisa Harddrive",
      "character": "D"
    },
    {
      "name": "Lenny Linux",
      "character": "C (Young)"
    }
  ]
}

\end{verbatim}

With Json Pointer specification, information can be retrieved as 

\verb+ "actors\1\name" + 

In reality this is not very elegant...

See some valid criticism at \href{http://susanpotter.net/blogs/software/2011/07/why-json-pointer-falls-short/}{json pointer vs xpath}. Although there are valid arguments for using an xpath type of arguments, I am not convinced people are willing 
to turn the clock back and start using xpath.

See RFC 6901 \url{https://tools.ietf.org/html/rfc6901}

