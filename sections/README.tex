\chapter{Introduction}

The Go package phd provides useful routines for manipulating documents and outputting them in
various formats. 

It also provides a tempalting mechanism for LaTeX files, as well as a package manager for these 
files.

\section{What GoPhd Can do for you?}

GoPhd is an AI that helps you develop printed documents from all sorts of sources. All ypu need is mostly
to type your document.

- Provide scaffolding to create the disk structure of complex documents
- Saves your documents to any of popular cloud stores (AWS, Google Drive, Microsoft)
- Automatically produce a static website for your document (similar to gitbooks)
- Version Control via Git

All you need is to type your document

```
gophd init "mybook"

```

```
gophd edit "mybook"

```

```
gophd publish "mybook"

```



\section{Prerequisites}

- You need to have a working Go installation
- You need to have a TeX distribution on your machine such as TeXLive or MikTeX
- If you are to use any of the many font utilities, you will need numerous fonts.
- Extensive theming engine

\section{Installation}

There is a batch file for windows INSTALL.BAT that helps with the installation of the tool and some necessary fonts, that you might not find easily.

\section{CLI}
  \begin{verbatim}
   phd directory  
   phd new document_name  creates a directory tree

   Produces a pdf from a set of documents in directory

		├───config
		├───thesis.tex
		├───thesis
			├─── main.tex or mydocument.tex
			├─── config.toml
		    └─── chapter01.tex
		    └─── chapter02.tex
		    └─── chapter02.tex
		    └─── ...
		    └─── graphics
			    ├───logo.png
				└───machine.png
				├───amato.png
			    └───amato.jpg
			    └───amato.jpeg
		├───themes
				├───book
				├───article
				├───report
				├───thesis
				├───exam
				├───lecturenotes
				├───proceedings
					├───template
					├───license
					├───partials
		    		└───specials
		├───temp
	\end{verbatim}	

\section{Conventions}

If you creating a document out of a list of other files, it is preferable to keep the in a tree directory as shown above.

There is always a site wide configuration folder `config` with settings for various tasks, explained a bit later. phd looks up first in the directory it is operated from for site-wise configuration file. This is used to provide defaults, if no configuration is found
in the document directory. If none is found then it used build-in defaults.

*gophd* can be embedded in other programs.

Examples can be found in the examples directory.






